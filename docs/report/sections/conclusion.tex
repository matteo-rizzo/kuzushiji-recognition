% !TeX spellcheck = en_US
\section{CONCLUSIONS}
\label{sec:conclusions}

The objective of the project was accurately recognizing handwritten cursive Kuzushiji characters in ancient historical documents, which has been accomplished via a region-based object recognition approach. The proposed model featured a CenterNet like detector followed by a CNN classifier, and achieved satisfying performances, with an overall Kaggle score of $0.797$ on the public test set.\\
On the other hand, our tests with Yolo v2, did not produce a usable model.\\ Looking at the results in tables \ref{tab:finaltests} and \ref{tab:classres} we can conclude that data augmentation did not significantly impacted the performance of the classifier, although this model scored much better on the macro $F_1$ score and got a more balanced precision and recall score too. The lower overall score may be due to insufficient training, or to the introduction of too much noise through the augmentation procedure. On the contrary instead, the \textit{tiling} approach, positively influenced the prediction score of both our detectors and the Kaggle model. \\
Many improvements to our work could have been tested. For example, a more tailored tiling and pre-processing would have probably resulted in a better prediction accuracy. Furthermore it could have been worth to try different encoding and decoding architectures for the detection model, like VGG and Hourglass, which are also referenced in the CenterNet paper. Using a totally different approach, a Mask R-CNN could probably be used to detect the characters, since it can generate a pixel-wise mask that seems suitable to recognize characters shapes. The usage of a recurrent model to add context information on each character classification, could have resulted in an improvement as well. Indeed, like reported on the dataset description of the competition, some Kuzushiji characters are very similar, differing just for small details, and their meaning can be more easily inferred from context than by looking at the single character itself.
